\chapter{System Testing}
\textbf{Software Testing}
\\\hspace*{0.82cm}Software testing is the process used to assess the quality of computer software. Software testing is an empirical 
technical investigation conducted to provide stakeholders with information about the quality of the product or service under test, 
with respect to the context in which it is intended to operate. this includes, but is not limited to, the process of executing a 
program or application with the intent of finding software bugs.\\[0.5cm]
\hspace*{0.82cm}There are many approaches to software testing. Reviews, walkthroughs or inspection are considered as static testing, 
whereas actually running the program with a given set of test cases in a given development stage is referred to as dynamic 
testing.\\[0.5cm]
Software testing is used in association with verification and validation:
\begin{itemize}
 \item \textbf{Verification: }Have we built the software right (i.e., does it match the specification)?
 \item \textbf{Validation: }Have we built the right software (i.e., is this what the customer wants?)
\end{itemize}

\hspace*{0.82cm}Software testing can be done by software testier. until the 1950s the term software tester was used generally, 
but later it was also seen as a separate profession. Regarding the periods and the different goals in software testing there 
have been established different roles: test lead/manager , tester, test designer, test automater/automation developer, and test 
administrator.\\[0.5cm]
\newpage
\textbf{Test plan:}\\
\hspace*{0.82cm}It is a systematic approach to testing a system such as a machine or software. The plan typically contains a detailed understanding of what the eventual
workflow will be.\\[0.5cm] 
\textbf{Test bed:}\\
\hspace*{0.82cm}It is a platform for experimentation for large development projects. Test beds allow for rigorous, transparent and 
replicable testing of scientific theories, computational tools, and other new technologies.\\[0.5cm]
\textbf{Test environment:}\\
\hspace*{0.82cm}In software, the hardware and software requirements are known as the test bed. This is also known as the test 
environment.\\[0.5cm]
\textbf{Scenario testing:}\\
\hspace*{0.82cm}Scenario testing is a software testing activity that uses scenario tests, or simply scenarios, which are based on 
a hypothetical story to help a person think through a complex problem or system. they can be as simple as a diagram for a testing 
environment or they could be a description written in prose.\\[0.5cm]
\textbf{Test case:}\\
\hspace*{0.82cm}Test case in software engineering is a set of conditions or variables under which a tester will determine if a 
requirement or use case upon an application is partially or full satisfied. It may take many test cases to determine that a 
requirement fully satisfied. Test cases are often incorrectly referred to as test scripts. Test scripts are lines of code used 
mainly in automation tools. written test cases are usually collected into test suites.\\[0.5cm]
\textbf{Unit testing:}\\
\hspace*{0.82cm}Unit testing tests the minimal software component or module. each unit (basic component) of the software is tested 
to verify that the detailed design for the unit has been correctly implemented. In an object oriented environment, this is usually 
at the class level, and the minimal unit tests include the constructors and destructors.\\[0.5cm]
\hspace*{0.82cm}Integration testing exposes defects in the interfaces and interaction between integrated components (modules). 
Progressively larger groups of tested software components corresponding to elements of the architectural design are integrated and 
testing until the software works as a system. System testing tests a completely integrated system to verify that it meets its 
requirements. System integration testing verifies that a system is integrated to any external or third party systems defined in 
the system requirements.\\[0.5cm]
\textbf{Black box testing:}\\
\hspace*{0.82cm}Black box testing treats the software as a black box without any understanding of internal behavior. it aims to 
test the functionality according to the requirements. Thus the tester inputs data and only sees the output from the test object. 
This level of testing usually requires through test cases to be provided to the tester who then can simply verify that for a given 
input, the output value (or behavior), is the same as the expected value specified in the test case. Black box testing methods 
include: equivalence partitioning , boundary value analysis, all - pairs testing, fuzz testing , model-based testing, traceability 
matrix etc.\\[0.5cm]
\textbf{White box testing:}\\
\hspace*{0.82cm}White box testing, however is when the tester has access to the internal data structures, code, and algorithms. 
White box testing methods include creating tests to satisfy some code coverage criteria. For Example, the test designer can 
create tests to cause all statements in the program to be executed at least once. Other examples of white box testing are mutation 
testing and fault injection methods. White box testing includes all static testing.\\[0.5cm]
\textbf{Alpha testing:}\\
Alpha testing is simulated or actual operational testing by potential users customers or an independent test team at the developer's 
site. Alpha testing is often employed for off-the-shelf software as a form of internal acceptance testing. Before the software goes 
into beta testing.\\[0.5cm]
\textbf{Beta testing:}\\
\hspace*{0.82cm}Beta testing comes after alpha testing. Version of the software , known as beta versions , are released to a limited 
audience outside of the programming team. The software is released to groups of people so that further testing can ensure the product 
has few faults or bugs. Sometimes, beta version are made available to the open public to increase the feedback field to a maximal 
number of future users.\\[0.5cm]
\textbf{Regression testing:}\\
\hspace*{0.82cm}After modifying the software, either for a change in functionality or the fix defects, a regression test re-runs 
previously passing tests on the modified software to ensure that the modifications haven't unintentionally caused a regression of 
previous functionality. Regression testing can be performed at any or all of the above test levels. These regression tests are 
often automated.\\[0.5cm] 
\textbf{Sanity test:}\\
\hspace*{0.82cm}Sanity test is a basic test to quickly evaluate the validity of a claim or calculation. in mathematics, for example, 
when dividing by three or nine, verifying that the sum of the digits of the result is a multiple of 3 or 9 (casting out nines) 
respectively is a sanity test.\\[0.5cm]
\textbf{Smoke testing:}\\
\hspace*{0.82cm}Smoke testing is a term used in plumbing, woodwind repair, electronics and computer software development. It refers 
to the first test made after repairs or first assembly to provide some assurance that the system under tests will catastrophically 
fail. After a smoke test proves that the pipes will not leak, the keys seal properly, the circuit will not burn, or the software will 
not crash outright, the assemble is ready for more stressful testing.
\newpage
\section{Test Cases and Test Results}
\begin{longtable}{ | p{1cm} | p{3.5cm} | p{4cm} | p{4cm} | p{4cm} |}
      \hline
      \textbf{Test ID} & \textbf{Test Case Title} & \textbf{Test Condition} & \textbf{System Behavior} & \textbf{Expected Result}\\
      \hline
      T01 & Application Installed on Server & Check \newline application is installed or not on Server & 1. Application is not shown in list of installed packages\newline2. Application is shown in list of installed packages & 1. Application is not installed\newline2. Application is installed\\
      \hline
      T02 & Application Installed on Client & Check \newline application is installed or not on client &  1. Application is not shown in list of installed packages\newline2. Application is shown in list of installed packages  & 1. Application is not installed\newline2. Application is installed\\
      \hline
      T03 & Bluetooth is ON/OFF & Check bluetooth is on/off &  1. Bluetooth ON is shown by system\newline 2. Bluetooth OFF is shown by system & 1. Bluetooth is ON\newline 2. Bluetooth is OFF\\
      \hline
      T04 & WiFi is ON/OFF & Check WiFi is on/off & 1. WiFi ON is shown by system\newline 2. WiFi OFF is shown by system & 1. WiFi is ON\newline 2. WiFi is OFF\\
      \hline
      T05 & USB cable is connected & Check USB cable is connected or not & 1)Device is not shown by system\newline 2) Device is shown by system & 1)USB cable is connected\newline 2) USB cable is not connected\\
      \hline
      T06 & Check mobile is connected to server or not & Check for list of bluetooth devices connected to server & List of connected devices is shown on server & Name of device is shown in connected devices\\
      \hline
      T07 & Is mobile device visible & Check mobile visiblity is ON/OFF & 1) Mobile visibility is OFF\newline 2) Mobile visibility is ON & 1) Device is not shown by bluetooth\newline 2) Device is shown by bluetooth\\
      \hline
      T08 & Check mobile exist in list ofpaired devices & Check list of paired devices & 1) Mobile Device doesn't exist in listof paired devices\newline 2) Mobile Device exist in list of paired devices & 1) Add device in list of paired devices\newline 2) Start next process\\
      \hline
      T09&Enter correct passphrase&Check entered passphrase is correct&1) Entered passphrase is Wrong\newline 2) Entered passphrase is Correct&1) Device will not get added on Server\newline 2) Device will be added to Server\\
      \hline
      T10&Check mobile exist in list of paired devices&Check list of paired devices&1) Mobile Device doesn't exist in list of paired devices\newline 2) Mobile Device exist in list of paired devices&1) Add device in list of paired devices\newline 2) Start next process\\
      \hline
      T11&User account exist or not for server&Check for User Login&1) User account  Exist\newline 2) User account does not exist&1)Start next process\newline 2) Create new account\\
      \hline
      T12&Enter Username&Check username is correct or not1) Username is not correct\newline 2) Username is correct&1) Enter username again\newline 2) password is focused&\\
      \hline
      T13&Enter Password&Check password is correct or not&1) Password is not correct\newline 2) Password is correct&1) Enter password again\newline 2) user get verified\\
      \hline
      T14&Check for new data on mobile device&Check new data exist on mobile device or not&1) New data does not exist on mobile device\newline 2) New data exist on mobile device&1) Do nothing\newline 2) Start Synchronization\\
      \hline
\end{longtable}

\textbf{Note: Testing should be performed manually}