\chapter{Project Planning}

project planning model:

Development occurs as a succession of releases with increasing functionality. Testing and feedback on earch release is used in deciding requirements and improvements for next release. THere is no "maintenance"
phase = each version includes both problem fixes as well as new features. this may also include "re-engineering" - changing the design and implementation of existing functionality, for easier maintainability. 

We have followed a spiral model like approach for our project as it has a large no. of units. So, prototyping at each stage is necessary. 
The spiral model is a software development process combining elements of both design and prototyping-in-stages, in an effort to combine advantages of top=down and bottom-up concepts. 
Also known as the spiral lifecycle model it is a systems development method(SDM) used in information technology (IT). This model of development combines the features of the prototyping model and the waterfall model. 
The spiral model is intended for large expensive and complicated projects. 
The steps in the spiral model can be generalized as follows:

the steps in the spiral model can be feneralized as follows:
1. the new system requirements are defined in as much detail as possible. A preliminary design is created for the new system. 
2. A first prototype of the new system is constructed from the preliminary design. This is usually a scaled down system, and represents an approximation of the characteristics of the final product. 

3. A second prototype is evolved by a fourfold procedure:
1.Evaluating the first prototype in terms of its strengths, weaknesses, and risks;
2. Defining the requirements of the second prototype
3. Planning and designing the second prototype 
4. construction and testing he second prototype 

4. At the customer's option , the entire project cna be aborted if the risk is deemed too great. Risk factors might involve development cost overruns, operating- cost miscalculation, or any other factor that couldint he customer's judgement, result in a less than satisfactory final product.

5. the existing prototype is evaluated in teh same manner as was the previous prototype, and , if necessary another prototype is developed from it according to the fourgold procedure outlined above. 
The preceding steps are iterated until the customer is satisfied that the refined prototype represents the final product desired. 

6. the final system is constructed , based on the refined prototype. 
7. The final system is thoroughly evaluated and tested. Routine maintenance is carried out on a continuing basis to prevent large-scale failures and to minimize downtime. 

Advantages: 
  Estimates(i.e. budget, schedule, etc) become more realistic as word progresses, because important issues are discovered earlier. 

it is more able to cope witht he (nearly inevitable) changes that software development generally entails. 

Software engineers (who can get restless with protracted design processes) can get their hands in and start working on a project earlier. 





Estimation and efforts: 
Technical : as all the technical knowledge required for developing the system is available through books, it is technicaly feasible. 
Economical: As no extra investment is required for hte imlementation of the system, it is sconomically feasible. 

Time requirements: 400 hours. 

Project schedule: 

  software project scheduling is an activity that distributes estimated effort accross the planned project duration. By allocation the effort to specific software engineering task like all other areas of software engineering, number of basic principles guide software project scheduling`` 

1. compartmentalization: 
it is accomplished by  decomposing both the product and the process into manageable activities and tasks. This is called work structure break down. 

2. Interdependency: 
SOme task must occur in a sequence while some must occur in parallel. The interdependencies of each compartmentalized act mustb be determined. 

3. time allocation: 
Each task to be scheduled must be allocated some number of work units. 
4. Effort Validation:
  Every project has a defined no. o fteam members and project leader should allocate the number of people that are ssheduled at any given time for a particular task. 

5. defined responsibilities: 
  every task  that is scheduled should be assinged to a specific team member. 

6. defined outcomes: 
Every task that is scheduled should have a defined outcome , normally a work product. Work product is often combined in deliverables. 

7. Defined milestones: 
Every task or group of task hsould be associated with project milestones.A milestone is accomplished when one or more work product has been reviewed for quality and has been approved.
Our preojct was schudeled to follow spiral model. We planned the project in following way: 
We spent the first two months i.e. July and August in Requirements Gathering and Surveying for software available and those to used for the project. Once we divided on it out next task was to design the front end of the system.

this took time as we had to first study the languages. After designing the front end we then decided to make the server module. In this we had to first learn how to use Servlets and their working. After this we tooks a long time to implement the interface of the front end and the server end. The month of april was dedicated to testing of the full 
system. Bugs were removed and testing was again performed in a spiral manner as spicified before. The final overall system was completed in the last week of April. 

Time ine Chart: 

timeline chart are user items that create a chart where a series of events are arranged on a bar graph . each event can be single point in time or a date range. the contents of a 
timeline chart are defined in groups of events. Each group is considered a timeline entry. Each group is assigned a title, a color, and other properties that vary by the timeline entry type. 

 




