\begin{center}
\thispagestyle{empty}
\vspace{2cm}
\LARGE{\textbf{ABSTRACT}}\\[1.0cm]
\end{center}
\thispagestyle{empty}
\hspace*{0.82cm}\large{This project aims at developing the Synchronization Manager for Android based devices. 
Since there isn't any generic Synchronization Manger for android which works on all platforms (Linux, Mac and Windows); 
this would help lots of people who work on two or more platforms. This Synchronization Manager would be divided into two 
parts as server and client.  The server would reside on PC and client would be on Android Device. Client will gather 
information which is to be synchronized.\\[0.3cm]}
\hspace*{0.82cm}\large{The current scenario insists users to send information to be backed up to google servers. 
Once this information is present with google, they have the right to sell/use that data in any way they want 
according to their “Terms of Service”. This hampers the users’ privacy.\\[0.3cm]}
\hspace*{0.82cm}\large{The primary purpose of this is to have offline synchronization. 
At later stages this can be extended to have online synchronization with services such as Ubuntu One for Linux.\\[0.3cm]}
\hspace*{0.82cm}\large{For offline syncing USB and Bluetooth may be used. If possible Wi-Fi can also be used. 
For communicating between server and client SyncML (Synchronization Markup Language) would be used.\\[0.3cm]}
\hspace*{0.82cm}\large{Data synchronization solution provides us a complete set of data security and data 
recovery tools to mop up with many problems. It also provides users and wireless carriers with a simpler, 
more connected means to accessing digital world.\\[0.3cm]}
\hspace*{0.82cm}\large{It ensure us about our valuable data be always reliable, protected and transferable. 
Its transparent and unique approach seamlessly synchronizes our data, keep backup, and manipulate any 
data accumulated on our mobile to main or central server. This facilitates different events in case of 
lost, upgrade or stolen mobile phone. Users of android mobile devices often need to synchronize their 
mobile devices with desktop computers.\\[0.3cm]}
\textbf{Keywords: }Data Synchronization, SyncML